\documentclass[ignorenonframetext,10pt,aspectratio=169]{beamer}

\usepackage{umut}
\usepackage{umuttr}
\usepackage{usynsem}
\usepackage[utf8]{inputenc}
\usepackage{uling}
\usepackage{natbib,unatbib}
\usepackage{linguex}
         \renewcommand{\refdash}{}
\usepackage{ubeamer}
\usepackage{verbatim}
\usepackage{adjustbox}
\usepackage{fancyvrb}

\usepackage{tikz-qtree}
\usetikzlibrary{er,positioning}

\title{Binding}
\author{\  \\  {\it Based on Koeneman \& Zeiljstra (2017)} \\ \vspace{20pt} Umut \"Ozge\\  }

\date{COGS 532: Theoretical Linguistics\\ METU, Informatics}

\begin{document}

\begin{frame}\frametitle{}
\thispagestyle{empty}
\maketitle
\end{frame}


\begin{frame}[t,plain]{Reflexives}


\ex. 
		\a. I\ind{i} like myself\ind{i}.
		\b. Mary\ind{i} sees herself\ind{i} in the mirror.
		\b. [Peter and Bill]\ind{i} excused themselves\ind{i}. 





\end{frame}

\begin{frame}[t,plain]{Pronouns}

\ex.
		\a. John\ind{i} told Mary\ind{j} that she\ind{j} can call him\ind{i} at 7pm.
		\b. They\ind{i} didn't tell John\ind{j} that he\ind{j} had misrepresented them\ind{i}.

\ex.
		\a. Mary called me. 
		\b. They said I was wrong.

\ex. Every boy who knows his house can direct you to it.

\end{frame}

\begin{frame}[t,plain]{The notion}

		\vspace{40pt}

		Binding is a semantic phenomenon, namely \alert{coreference}, that obeys syntactic constraints.
	
\end{frame}

\begin{frame}[t,plain]{Syntactic constraints}
		\framesubtitle{Reflexives}

		Reflexives \alert{must} be bound: 

		\ex. \a. *You like myself.
		\b. *John sees herself in the mirror.
		\b. *Peter knows each other.



\end{frame}

\begin{frame}[t,plain]{Syntactic constraints}
		\framesubtitle{Domain of binding}

		\ex.  John said that Peter thought that Harry blamed himself.

		\vspace{40pt}

		\ex.  John said that Peter thought that Harry blamed him.

\end{frame}

\begin{frame}[t,plain]{The generalization}

		\vspace{40pt}

		\colb{Principle A:}\\	
		A reflexive \alert{must} be bound by a nearby antecedent.

\bigskip

		\colb{Principle B:}\\
		A non-reflexive pronoun \alert{cannot} be bound by a nearby antecedent. 


\end{frame}

\begin{frame}[t,plain]{}

\end{frame}

\begin{frame}[t,plain]{}

\end{frame}

\begin{frame}[t,plain]{}

\end{frame}

\end{document}
