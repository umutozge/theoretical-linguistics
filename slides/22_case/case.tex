\documentclass[ignorenonframetext,10pt,aspectratio=169]{beamer}

\usepackage{umut}
\usepackage{umuttr}
\usepackage{usynsem}
\usepackage[utf8]{inputenc}
\usepackage{uling}
\usepackage{natbib,unatbib}
\usepackage{linguex}
         \renewcommand{\refdash}{}
\usepackage{ubeamer}
\usepackage{verbatim}

\usepackage{fancyvrb}

\usepackage{tikz-qtree}
\usetikzlibrary{er,positioning}

\title{Case}
\author{\  \\  {\it Partly based on Koeneman \& Zeiljstra (2017)} \\ \vspace{20pt} Umut \"Ozge\\  }

\date{COGS 532: Theoretical Linguistics\\ METU, Informatics}

\begin{document}

\begin{frame}\frametitle{}
\thispagestyle{empty}
\maketitle
\end{frame}

\begin{frame}[t,plain]{Why a Case Theory?}

\ex. She loves her. 

\ex. *Her loves she.

\pause

\begin{itemize}
\item Why not handle it with $\Theta$-theory? 
\end{itemize}
\pause
\ex. She is loved.

\ex. I caused him to quit his job.

\end{frame}

\begin{frame}[t,plain]{}
\vspace{50pt}
Case is a property of syntactic environment.
\end{frame}

\begin{frame}[t,plain]{Case assignment: Accusative}
\begin{itemize}
\item Verbs and Prepositions assign accusative case.
\end{itemize}

\ex. \a. The book about her.
\b. The anger in him.

\vspace{30pt}

\ex. Julia's love of him. 

\end{frame}

\begin{frame}[t,plain]{Accusative and adjacency}

\ex.
\a. John very often believes him.
\b. *John  believes very often him.
\b. John believes him very often. 

\ex.
\a. John very often believes in him.
\b. John believes very often in him.
\b. John believes in him very often. 
\b. *John believes in very often him. 

\end{frame}

\begin{frame}[t,plain]{}

\end{frame}

\end{document}
