\documentclass[ignorenonframetext,10pt,aspectratio=169]{beamer}

\usepackage{umut}
\usepackage{umuttr}
\usepackage[utf8]{inputenc}
\usepackage{uling}
\usepackage{natbib,unatbib}
\usepackage{linguex}
         \renewcommand{\refdash}{}
\usepackage{ubeamer}
\usepackage{verbatim}

\usepackage{fancyvrb}

\usepackage{tikz-qtree}
\usetikzlibrary{er,positioning}

\title{Concepts and grammar}
\author{\  \\ \vspace{20pt} Umut \"Ozge\\  }

\date{COGS 532: Theoretical Linguistics\\ METU, Informatics}

\begin{document}

\begin{frame}\frametitle{}
\thispagestyle{empty}
\maketitle
\end{frame}

\begin{frame}[t,plain]
\begin{itemize}
\item \Verb+FEAR+ is a concept relating an \Verb+EXPERIENCER+ with a \Verb+STIMULUS+.

\item Syntacticizations:

\ex. \a. John fears spiders.
\b. Spiders frighten John.
\b. John is scared of spiders.

\item The \Verb+CAUSE+ of fear is syntacticized differently:

\ex. Bill frightened Sue by his malicious grin.

\item Compare the following syntacticizations of \Verb+SEARCH+:

\ex. John googled that strange word.

\end{itemize}
\end{frame}

\begin{frame}[t,plain]

\end{frame}

\begin{frame}[t,plain]

\end{frame}

\begin{frame}[t,plain]

\end{frame}

\begin{frame}[t,plain]

\end{frame}
\end{document}
