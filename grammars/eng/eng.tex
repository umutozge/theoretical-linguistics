\documentclass{article}



\usepackage[a4paper]{geometry}
\usepackage{fourier}

\usepackage{enumitem}



\usepackage{linguex}
\usepackage{amsmath}
\usepackage{etoolbox}
\AtBeginEnvironment{align}{\setcounter{equation}{0}}

\usepackage{mathtools}


% Tree drawing
\usepackage{tikz-qtree}
\usepackage{xargs}
\newcommandx{\tikztree}[3][2=25,3=35,usedefault]{%
\begin{tikzpicture}%
\tikzset{level distance=#2pt, sibling distance=#3pt}%
\tikzset{every tree node/.style={align=center,anchor=north}}%
#1
\end{tikzpicture}
}


% Custom commands

%% To have nicer math mode spacing between letters
\newcommand{\syn}[1]{\ensuremath{\mathrm{#1}}}
\newcommand{\syndef}[2]{\mathrm{#1}&\qquad :=& \mathrm{#2}}
\newcommand{\gmerge}[3]{\ensuremath{(\mathrm{#1},\mathrm{#2})\quad\Rightarrow\quad \mathrm{#3}}}
\newcommand{\un}[1]{\ensuremath{\prescript{u}{}{\mathrm{#1}}}}
\newcommand{\ux}[1]{\ensuremath{\prescript{x}{}{\mathrm{#1}}}}

\newcommand{\setcomp}[2]{\ensuremath{\{#1\,|\,#2\}}}

% Theorem style

\RequirePackage{amsthm}

\newtheoremstyle{grammar}%
{5pt} 		% Space above
{5pt} 		% Space below
{\rm} 		% Body font
{}	% Indent amount
{\bf}      % Theorem head font
{.}      % Punct. after theorem head
{.5em} 	% Space after theorem head
{} 			% Theorem head spec.


\theoremstyle{grammar}
\newtheorem{gclaim}{Claim}[section]
\newtheorem{gassumption}{Assumption}[section]
\newtheorem{glexicon}{Lexicon}[section]
\newtheorem{grule}{Rule}[section]
\newtheorem{gconvention}{Convention}[section]
\newtheorem{gtheorem}[gclaim]{Theorem}
\newtheorem{gproblem}[gclaim]{Problem}
\newtheorem{gquestion}[gclaim]{Question}
\newtheorem{gobservation}[gclaim]{Observation}
\newtheorem{gdefinition}[gclaim]{Definition}
\newtheorem{gchallenge}[gclaim]{Challenge}
\newtheorem{gexercise}[gclaim]{Exercise}
\newtheorem{gexample}[gclaim]{Example}
\newtheorem{gcode}[gclaim]{Code}
\newtheorem{galgorithm}[gclaim]{Algorithm}


\title{An English fragment}
\author{COGS 532, Spring 2023}

\begin{document}
\maketitle


\section{The formalism}


We define the components of our formalism.

\begin{gdefinition}[Vocabulary] The vocabulary of the grammar formalism
	consists of the following sets:
	\begin{enumerate}[label=\roman*.,labelsep=.5em]
		\item a set of attribute symbols $A$;
		\item a set of basic value symbols $V^b$;
		\item a set of decorations  $D = \{u,x\}$; 
		\item the set of value symbols $V=\setcomp{v}{v\in V^b \text{ or
			} v=\prescript{\delta}{}{v'} \text{ for some } v'\in V^b
		\text{ and } \delta\in D}$
		\item a set of variables $X$;
	\end{enumerate}
\end{gdefinition}

\begin{gdefinition}[Feature map] A feature map of a grammar, designated $\mu$,
	is a function from $A$ to $\mathcal{P}(V)$.\footnote{$\mathcal{P}$ stands
	for the power set function.} It maps each attribute symbol to the set of
	possible values it can take.
\end{gdefinition}

% \begin{gdefinition}[Attribute-value pairs]
% 	The set of attribute-value pairs of a grammar is defined as
% 	\begin{align*}
% 		\setcomp{a:v}{a\in A \text{ and } v\in \mu(a)}
% 	\end{align*}
% \end{gdefinition}

\begin{gdefinition}[Feature structures]\label{dffs}
	A possible feature structure defined in a grammar is some set
	\begin{align*}
		\setcomp{a:v}{a\in A' \text{ and } v\in \mu(a)} \text{ for some } 
		A'\subseteq A
	\end{align*}
\end{gdefinition}

Note that Definition~\ref{dffs} guarantees that a feature structure can set a
value for an attribute at most once. Also, it rules out nested feature
structures, where the value of an attribute is also a feature structure.


\begin{gdefinition}[Merge]\label{dfmerge}
	The operation Merge takes as input two feature structures $F_1$ and
	$F_2$ and returns a feature structure $F_3$. The result is computed
	through the following steps:
	\begin{enumerate}[label=\roman*.,labelsep=.5em]
		\item For each feature $a:v$ in $F_1$, if you find a feature 
			$\un{a}:v$ in $F_2$ delete the latter, and vice versa.
		\item For each feature $a:v$ in $F_1$, if you find a feature
			$\ux{a}:v$ in $F_2$ delete both features, and vice
			versa. 
		\item If there are no conflicts in feature structures, merge
			them into a single feature structure. 
	\end{enumerate}
\end{gdefinition}

\section{Notational conventions}

\begin{gconvention}
	An attribute value pair $a:v$ is shortened to $a^v$.
\end{gconvention}


\begin{gconvention}
	When obvious, we omit the attribute. E.g.\ we write $\syn{N}$ instead of
	$\syn{Cat^N}$.
\end{gconvention}

\section{Features}

\begin{tabular}{lll}
	Attribute & Value             & Description\\ 
	\syn{Cat}	  & \syn{N,V,D,Adj,C,\ldots} & category\\
	\syn{Cmp}      & \syn{0,1,2,3}     & Number of missing complements\\
	\syn{Spc}      & \syn{+,-}         & Specified or not \\
	$\syn{\phi} $ \syn{0,1}    & Person, number\\ 

\end{tabular}

\section{Lexicon}

\begin{glexicon}[nominal]
\begin{align}
\syndef{book}{[N,\un{V}]}\\
\syndef{book}{[N,\un{Fin}]}\\
\syndef{the}{[D,\ux{N}]}\\
\syndef{a}{[D,\ux{N}]}\\
\syndef{John}{[D,N]}
\end{align}
\end{glexicon}


\begin{glexicon}[verbal]
\begin{align}
\syndef{sleep}{[V]}\\
\syndef{love}{[V,\ux{D}]}\\
\end{align}
\end{glexicon}

\begin{glexicon}[functional]
\begin{align}
				\syndef{-s}{[Fin^+,\ux{V},\ux{D}]}\\
	      \syndef{to}{[Fin^-,\ux{V}]}\\
\end{align}
\end{glexicon}


\ex. Mary loves John.

\tikztree{
	\Tree 
	[.{Mary loves John}\\\syn{[Fin^+]}
   {Mary}\\\syn{[D,\un{Fin^+}]}
	 [.{loves John}\\\syn{[Fin^+,\ux{D}]}
	 {-s}\\\syn{[Fin^+,\ux{V},\ux{D}]}
    [.{love John}\\\syn{[V]}
     {love}\\\syn{[V,\ux{D}]}
     {John}\\\syn{[D,\un{V}]}
     ]
    ]
   ]
}[35][40]

\end{document}
