\documentclass[ignorenonframetext,10pt,aspectratio=169]{beamer}

\usepackage{umut}
\usepackage{umuttr}
\usepackage[utf8]{inputenc}
\usepackage{uling}
\usepackage{natbib,unatbib}
\usepackage{linguex}
         \renewcommand{\refdash}{}
\usepackage{ubeamer}
\usepackage{verbatim}

\usepackage{fancyvrb}

\usepackage{tikz-qtree}
\usetikzlibrary{er,positioning}

\title{Categories and features}
\author{\  \\  {\it Based on Koeneman \& Zeiljstra (2017)} \\ \vspace{20pt} Umut \"Ozge\\  }

\date{COGS 532: Theoretical Linguistics\\ METU, Informatics}

\begin{document}

\begin{frame}\frametitle{}
\thispagestyle{empty}
\maketitle
\end{frame}

\begin{frame}[t,plain]{Why have categories?}

\end{frame}

\begin{frame}[t,plain]{What distinguishes categories?}
\pause

\begin{itemize}
\item Form is totally, meaning is partially unreliable.

\ex. Everybody was \alert{dancing} well; John's \alert{dancing} was the best.

\end{itemize}
\end{frame}

\begin{frame}[t,plain]{The linguistic sign}

\end{frame}

\begin{frame}[t,plain]{Substitution test}

If two elements $X$ and $Y$ share the same syntactic features then every grammatical sentence that contains $X$ remains grammatical when $X$ is replaced by $Y$.
\pause
\ex. John saw \paradigm{\text{Mary}\\ \text{the car}\\ \text{every spy holding a martini glass}}. 

\end{frame}


\begin{frame}[t,plain]{Subfeatures}
\ex. John saw \paradigm{\text{the car} \\ \text{the cars}}.

\end{frame}

\begin{frame}[t,plain]{Beyond surface form}

\ex. John saw \paradigm{\text{the car}\\ \text{a car} \\ \text{*car}\\ \text{the cars} \\ \text{*a cars}\\ \text{cars} \\ \text{blood} \\ \text{Mary} \\ \text{her}}.

\end{frame}

\begin{frame}[t,plain,label=verbalf]{Verbal features}

\ex. John \paradigm{\text{\colb{dances}}\\ \text{\colb{danced}} \\ \text{wants \colb{to dance}} \\ \text{has \colb{danced}} \\ \text{is \colb{dancing}}}

\end{frame}

\againframe[t,plain]{verbalf}

\begin{frame}[t,plain]

\end{frame}
\end{document}
