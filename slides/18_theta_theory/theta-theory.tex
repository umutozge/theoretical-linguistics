
\documentclass[ignorenonframetext,10pt,aspectratio=169]{beamer}

\usepackage{umut}
\usepackage{umuttr}
\usepackage{usynsem}
\usepackage[utf8]{inputenc}
\usepackage{uling}
\usepackage{natbib,unatbib}
\usepackage{linguex}
         \renewcommand{\refdash}{}
\usepackage{ubeamer}
\usepackage{verbatim}

\usepackage{fancyvrb}

\usepackage{tikz-qtree}
\usetikzlibrary{er,positioning}

\title{Theta theory}
\author{\  \\  {\it Partly based on Koeneman \& Zeiljstra (2017)} \\ \vspace{20pt} Umut \"Ozge\\  }

	\date{COGS 532: Theoretical Linguistics\\ METU, Informatics}

\begin{document}

\begin{frame}\frametitle{}
\thispagestyle{empty}
\maketitle
\end{frame}

\begin{frame}[t,plain]{}

\end{frame}

\begin{frame}[t,plain]{Syntax-Semantics}

\begin{itemize}
\item \alert{Argument} is a semantic notion.
\item \alert{Complement},  \alert{direct object}, \alert{adjunct} etc.\ are syntactic notions.
\item The interface is regulated by $\Theta$-theory (and Linking Theory).
\end{itemize}

\end{frame}

\begin{frame}[t,plain]{$\Theta$-criterion}
\begin{itemize}
\item Verbs are lexically specified to assign certain $\Theta$-roles to certain syntactic ``slots''.
\item $\Theta$-roles ans slots are in one-one correspondence.
\end{itemize}

\ex. Mary\hspace{40pt} killed\hspace{40pt} Bill. 

\medskip
\ex. John\hspace{40pt} gave\hspace{40pt} Mary\hspace{40pt} the book. 

\medskip
\ex. John\hspace{40pt} gave\hspace{40pt} the book\hspace{40pt} to Mary.

\end{frame}

\begin{frame}[t,plain]{Non-$\Theta$-marked semantic information}

\ex. John was reading the book.

\ex. John read the whole night.

\ex. John was reading.

\end{frame}

\begin{frame}[t,plain]{Subjects and Agents}
\ex.
\a. John sleeps.
\b. John snores.
\b. John walks.

\ex. \a. Bill fell.
\b. John died.
\b. The glass broke.

\end{frame}

\begin{frame}[t,plain]{$\Theta$-hierarchy}

\begin{itemize}
\item E.g.\ If a verb assigns AGENT and THEME (or PATIENT), it is the AGENT that appears as subject.
\pause
\item AGENT > RECIPIENT > THEME/PATIENT > GOAL
\end{itemize}


\end{frame}

\begin{frame}[t,plain]{Can we predict the optionality of $\Theta$-assignment?}

\ex. \a.John ate his dinner.\\ \b. John ate.

\ex. \a.John asked a question.\\ \b. *John asked.

\end{frame}

\begin{frame}[t,plain]{}

\end{frame}

\end{document}
